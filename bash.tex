\documentclass[10pt,landscape]{article}
\usepackage{multicol}
\usepackage{calc}
\usepackage{ifthen}
\usepackage[landscape]{geometry}
\usepackage{hyperref}

% To make this come out properly in landscape mode, do one of the following
% 1.
%  pdflatex latexsheet.tex
%
% 2.
%  latex latexsheet.tex
%  dvips -P pdf  -t landscape latexsheet.dvi
%  ps2pdf latexsheet.ps


% If you're reading this, be prepared for confusion.  Making this was
% a learning experience for me, and it shows.  Much of the placement
% was hacked in; if you make it better, let me know...


% 2008-04
% Changed page margin code to use the geometry package. Also added code for
% conditional page margins, depending on paper size. Thanks to Uwe Ziegenhagen
% for the suggestions.

% 2006-08
% Made changes based on suggestions from Gene Cooperman. <gene at ccs.neu.edu>


% To Do:
% \listoffigures \listoftables
% \setcounter{secnumdepth}{0}


% This sets page margins to .5 inch if using letter paper, and to 1cm
% if using A4 paper. (This probably isn't strictly necessary.)
% If using another size paper, use default 1cm margins.
\ifthenelse{\lengthtest { \paperwidth = 11in}}
{ \geometry{top=.5in,left=.5in,right=.5in,bottom=.5in} }
{\ifthenelse{ \lengthtest{ \paperwidth = 297mm}}
	{\geometry{top=1cm,left=1cm,right=1cm,bottom=1cm} }
	{\geometry{top=1cm,left=1cm,right=1cm,bottom=1cm} }
}

% Turn off header and footer
\pagestyle{empty}


% Redefine section commands to use less space
\makeatletter
\renewcommand{\section}{\@startsection{section}{1}{0mm}%
	{-1ex plus -.5ex minus -.2ex}%
	{0.5ex plus .2ex}%x
	{\normalfont\large\bfseries}}
\renewcommand{\subsection}{\@startsection{subsection}{2}{0mm}%
	{-1explus -.5ex minus -.2ex}%
	{0.5ex plus .2ex}%
	{\normalfont\normalsize\bfseries}}
\renewcommand{\subsubsection}{\@startsection{subsubsection}{3}{0mm}%
	{-1ex plus -.5ex minus -.2ex}%
	{1ex plus .2ex}%
	{\normalfont\small\bfseries}}
\makeatother

% Define BibTeX command
\def\BibTeX{{\rm B\kern-.05em{\sc i\kern-.025em b}\kern-.08em
		T\kern-.1667em\lower.7ex\hbox{E}\kern-.125emX}}

% Don't print section numbers
\setcounter{secnumdepth}{0}


\setlength{\parindent}{0pt}
\setlength{\parskip}{0pt plus 0.5ex}

\newcommand{\commandfmt}[1]{\ttfamily\bfseries #1}
\newcommand{\command}[2]{\commandfmt{#1} & #2 \\}
\newenvironment{commands}
{\settowidth{\MyLen}{\commandfmt{Backspace..}}
	\begin{tabular}{@{}p{\the\MyLen}%
			@{}p{\linewidth-\the\MyLen}@{}}}{\end{tabular}}

% -----------------------------------------------------------------------
\newlength{\MyLen}
\begin{document}
	
\raggedright
\footnotesize 
	\begin{multicols*}{3}
		
		
		% multicol parameters
		% These lengths are set only within the two main columns
		%\setlength{\columnseprule}{0.25pt}
		\setlength{\premulticols}{1pt}
		\setlength{\postmulticols}{1pt}
		\setlength{\multicolsep}{1pt}
		\setlength{\columnsep}{2pt}
		
		\begin{center}
			\Large{\textbf{bash CLI Cheatsheet}} \\
		\end{center}
		
\section{Notation}
\begin{commands}
  \command{C-<char>}{Hold Control key and press the character.}
  \command{M-<char>}{Hold the Meta key and press the character.}
\end{commands}
	
\section{Commands for Moving}
\begin{commands}
  \command{C-a}{Move cursor to the start of the current line.}
  \command{C-e}{Move cursor to the end of the line.}
  \command{C-f}{Move cursor forward a character.}
  \command{C-b}{Move cursor back a character.}
  \command{M-f}{Move cursor forward to the end of the next word.}
  \command{M-b}{Move cursor back to the start of the current or previous word.}
  \command{C-l}{Clear the screen leaving the current line at the top of the screen.}
\end{commands}

\section{Commands for Changing Text}
\begin{commands}
  \command{DEL or Backspace}{Delete the character to the left of the cursor}
  \command{C-d}{Delete the character at point.}
  \command{C-v}{Add the next character typed to the line verbatim.}
  \command{C-v TAB}{Insert a tab character.}
  \command{C-t}{Transpose two characters.}
  \command{M-t}{Transpose two words.}
  \command{M-u}{Uppercase the current (or following) word.  With a negative argument, uppercase the previous word, but do not move point.}
  \command{M-l}{Lowercase the current (or following) word.  With a negative argument, lowercase the previous word, but do not move point.}
  \command{M-c}{Capitalize the current (or following) word.  With a negative argument, capitalize the previous word, but do not move point.}
\end{commands}

\section{Killing Commands}
\begin{commands}
  \command{C-k}{Kill the text from the current cursor position to the end of the line}
  \command{M-d}{Kill from the cursor to the end of the current word, or, if between words, to the end of the next word. Word boundaries are the same as those used by M-f}
  \command{M-DEL}{Kill from the cursor the start of the current word, or, if between words, to the start of the previous word. Word boundaries are the same as those used by M-b}
  \command{C-w}{Kill from the cursor to the previous whitespace. This is different than M-DEL because the word boundaries differ}

%Here is how to yank the text back into the line. Yanking means to copy the most-recently-killed text from the kill buffer. \\

  \command{C-y}{Yank the most recently killed text back into the buffer at the cursor}
  \command{M-y}{Rotate the kill-ring, and yank the new top. You can only do this if the prior command is C-y or M-y}	
\end{commands}
\end{multicols*}
\end{document}